\documentclass{jsarticle}
\usepackage{amssymb,amsmath}
\usepackage{bcprules}
\usepackage{multicol}


\title{The Definition of Snail}
\author{letexpr}

\newcommand{\bnfdef}{::=}
\newcommand{\bs}{\boldsymbol}
\newlength{\len}
\settowidth{\len}{$\bnfdef$}
\newcommand{\bnfor}{\makebox[\len]{$|$}}

\DeclareMathSymbol{\mhyph}{\mathalpha}{operators}{`-}

\begin{document}
\maketitle

\section{Snailの構文定義}

EBNF記法を用いてSnailの具象構文を以下に示す.

\begin{eqnarray*}
  toplevel &\bnfdef& let\ [ rec ]\ var\ \{ var\ [\ \colon\ \langle type \rangle ] \} \colon \langle type \rangle = \langle term \rangle\ \{ \langle mutual \mhyph recursion \mhyph top \mhyph let \rangle \} \\
  &\bnfor& typedef\ cons\ \{ var \} = [\ |\ ]\ \{ \langle type \mhyph dec \rangle\ |\ \}\ \langle type \mhyph dec \rangle\ \{ \langle mutual \mhyph recursion \mhyph type \rangle \} \\ \\
  mutual \mhyph recursion \mhyph type &\bnfdef& and\ cons\ \{ var \} = [\ |\ ]\ \{ \langle type \mhyph dec \rangle\  |\ \}\ \langle type \mhyph dec \rangle \\ \\
  mutual \mhyph recursion \mhyph top \mhyph let &\bnfdef& and\ var\ \{ var\ [\ \colon\ \langle type \rangle ] \}\ \colon\ \langle type \rangle = \langle term \rangle \\ \\
  type \mhyph dec &\bnfdef& cons\ [ of\ \langle type \rangle ] \\ \\
  type &\bnfdef& \langle type \rangle \rightarrow \langle type \rangle \\
  &\bnfor& !\ '['\ \langle expmod \rangle\ ']'\ '\{'\ \langle type \rangle\ '\}' \\
  &\bnfor& \langle simple \mhyph type \rangle \\
  &\bnfor& \langle type \rangle\ \langle simple \mhyph type \rangle \\ \\
  expmod &\bnfdef& int \\
  &\bnfor& \infty \\ \\
  simple \mhyph type &\bnfdef& '('\ \langle type \rangle\ ')' \\
  &\bnfor& var \\
  &\bnfor& cons \\
  &\bnfor& () \\ \\
  pattern &\bnfdef& \langle simple \mhyph pattern \rangle \\
  &\bnfor& \langle pattern \rangle\ \langle simple \mhyph pattern \rangle \\
  &\bnfor& \langle simple \mhyph pattern \rangle\ binop\ \langle simple \mhyph pattern \rangle \\
\end{eqnarray*}

\newpage

\begin{eqnarray*}
  simple \mhyph pattern &\bnfdef& '('\ \langle pattern \rangle\ ')' \\
  &\bnfor& var \\
  &\bnfor& cons\ '['\ \langle simple \mhyph pattern \rangle\ ']' \\
  &\bnfor& [\ ] \\
  &\bnfor& \_ \\ \\
  mutual \mhyph recursion \mhyph let &\bnfdef& and\ var\ \{ var\ [\ \colon\ \langle type \rangle ] \}\ \colon\ \langle type \rangle = \langle term \rangle \\ \\
  term &\bnfdef& \langle simple \mhyph term \rangle \\
  &\bnfor& \langle term \rangle\ \langle simple \mhyph term \rangle \\
  &\bnfor& let\ [ rec ]\ var\ \{ var\ [\ \colon\ \langle type \rangle ] \} \colon \langle type \rangle = \langle term \rangle\ \{ \langle mutual \mhyph recursion \mhyph let \rangle \}\ in\ \langle term \rangle \\
  &\bnfor& fun\ \{ var\ [\ \colon\ \langle type \rangle ] \}\ \rightarrow \langle term \rangle \\
  &\bnfor& match\ \langle term \rangle\ with\ [\ |\ ]\ \{ \langle pattern \rangle \rightarrow \langle term \rangle\ |\ \}\ \langle pattern \rangle \rightarrow \langle term \rangle\\
  &\bnfor& if\ \langle term \rangle\ then\ \langle term \rangle\ else\ \langle term \rangle \\ \\
  simple \mhyph term &\bnfdef& '('\ \langle term \rangle\ [\ \colon \langle type \rangle]\ ')'\\
  &\bnfor& !\ \langle term \rangle\\
  &\bnfor& int \\
  &\bnfor& float \\
  &\bnfor& string \\
  &\bnfor& bool \\
  &\bnfor& var \\
  &\bnfor& cons\ [ \langle simple \mhyph term \rangle ] \\
  &\bnfor& () \\
  &\bnfor& [\ ] \\
  &\bnfor& list
\end{eqnarray*}

終端記号の意味を以下のように定義する.

\begin{itemize}
  \item var\ 先頭が小文字で始まる文字列.
  \item cons\ 先頭が大文字で始まる文字列.
  \item list\ 組み込みリストの構文糖衣,[1,2,3]など.
  \item string\ 文字列リテラル.
  \item int\ 整数リテラル.
  \item float\ 小数リテラル.
  \item bool\ 真偽値リテラル.
  \item その他\ 予約語.
\end{itemize}

\newpage

\section{Snailの型システム}

Snailは次のような型付け規則を持つ.

\subsection{型付け規則}

\begin{multicols}{2}

  \infrule[Int]{}{
    \vdash int\ \colon Int
  }

  \infrule[Float]{}{
    \vdash float\ \colon Float
  }

  \infrule[String]{}{
    \vdash string\ \colon String
  }

  \infrule[Bool]{}{
    \vdash bool\ \colon Bool
  }

  \infrule[id]{}{
    x\ \colon A \vdash x\ \colon A
  }

  \infrule[der]{
    \Gamma , x\ \colon A \vdash e\ \colon B
  }{
    \Gamma , x\ \colon [A]_{1} \vdash e\ \colon B
  }

  \infrule[pr]{
  [ \Gamma ] \vdash e\ \colon B
  }{
  r * [\Gamma] \vdash\ !e\ \colon !_{r} B
  }

  \infrule[fun]{
    \Gamma , x\ \colon A \vdash e\ \colon B
  }{
    \Gamma \vdash fun\ x \rightarrow e\ \colon A \multimap B
  }

  \infrule[fun-exp]{
  \Gamma , x\ \colon [ A ]_{r} \vdash e\ \colon B
  }{
  \Gamma \vdash fun\ ( !x\ \colon !_{r} A ) \rightarrow e\ \colon\ !_{r} A \multimap B
  }

  \infrule[app]{
    \Gamma \vdash e\ \colon A \multimap B \hspace{15pt} \Delta \vdash e'\ \colon A
  }{
    \Gamma + \Delta \vdash e\ e'\ \colon B
  }

  \infrule[if]{
    \Gamma \vdash e\ \colon Bool \hspace{15pt} \Delta \vdash e_1\ \colon A \hspace{15pt} \Delta \vdash e_2\ \colon A
  }{
    \Gamma + \Delta \vdash if\ e\ then\ e_1\ else\ e_2\ \colon A
  }

  \infrule[let]{
    \Gamma \vdash e\ \colon A \hspace{15pt} \Delta , x\ \colon A \vdash e'\ \colon B
  }{
    \Gamma + \Delta \vdash let\ x = e\ in\ e'\ \colon B
  }

  \infrule[let-exp]{
  \Gamma \vdash e\ \colon !_{r} A \hspace{15pt} \Delta , x\ \colon [ A ]_{r} \vdash e'\ \colon B
  }{
  \Gamma + \Delta \vdash let\ !x = e\ in\ e'\ \colon B
  }

  \infrule[let-rec]{
  [ \Gamma ] , x\ \colon [A]_{p} \vdash e\ \colon\ A \hspace{15pt} \Delta , x\ \colon\ [ A ]_{\infty} \vdash e'\ \colon\ B
  }{
  \infty * [ \Gamma ] + \Delta \vdash let\ rec\ x = e\ in\ e'\ \colon B
  }

  \infrule[sub]{
    \Delta \vdash e\ \colon\ B \hspace{10pt} \Gamma <:\Delta
  }{
    \Gamma , \varTheta \vdash e\ \colon B
  }

\end{multicols}

\subsection{部分型付け規則}

\begin{multicols}{2}
  \infrule[o-i]{}{
    A <: A
  }

  \infrule[o-b]{
  A <: B \hspace{10pt} q \preceq p
  }{
  !_{p} A <: !_{q} B
  }

  \infrule[o-l]{
    A' <: A \hspace{10pt} B <: B'
  }{
    A \multimap B <: A' \multimap B'
  }

  \infrule[o-d]{
  A <: B \hspace{10pt} q \preceq p
  }{
  [ A ]_{p} <: [ B ]_{q}
  }

  \infrule[o-ic]{}{
    \Gamma <: \Gamma
  }

  \infrule[o-c]{
    \Gamma <: \Delta \hspace{10pt} A <: B
  }{
    \Gamma , x\ \colon B <: \Delta , x\ \colon A
  }
\end{multicols}

\section{参考文献}



\end{document}